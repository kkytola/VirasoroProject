\section{The basic bosonic Sugawara construction}%
\label{sec:sugawara}

Throughout this section, let $\bbk$ be a field of characteristic zero.


If a vector space $V$ has a representation %$\rho \colon \hei \to \mathrm{End}(V)$
of the Heisenberg algebra on a vector space $V$,
where the central element $K$ (see Definition~\ref{def:HeisenbergBasis}),
acts as $\id_V$, then the basis elements $(J_k)_{k \in \bZ}$
(see Definition~\ref{def:HeisenbergBasis}) are linear operators $\Joper_k \colon V \to V$
satisfying the commutation relations
\begin{align*}%\tag{heiComm}\label{eq:heiComm}
  \mathrm{\tagHeiComm} \qquad
  [\Joper_k, \Joper_l]
  \; = \; \Joper_k \circ \Joper_l - \Joper_l \circ \Joper_k
  \; = \; k \, \delta_{k+l,0} \; \id_V .
\end{align*}
Below we will assume such operators being fixed, and satisfying furthermore the
local truncation condition on $V$: for any fixed $v \in V$ we have
$\Joper_k \, v = 0$ for $k \gg 0$, i.e.,
\begin{align*}%\tag{heiTrunc}\label{eq:heiTrunc}
  \mathrm{\tagHeiTrunc} \qquad
  \forall v \in V , \; \exists N, \; \forall k \ge N, \quad
  \Joper_k \, v = 0 .
\end{align*}

\begin{definition}[Normal ordering]
  \label{def:NormalOrdering}
  %\uses{}
  \lean{VirasoroProject.pairNO}
  \leanok
  For $k,l \in \bZ$, we denote the normal ordered product of the operators $\Joper_k$
  and $\Joper_l$ by
  \begin{align*}
    \normalOrder{\Joper_k \, \Joper_l} \; := \;
    \begin{cases}
      \Joper_k \circ \Joper_l & \text{ if } k \le l \\
      \Joper_l \circ \Joper_k & \text{ if } k \, > \, l .
    \end{cases}
  \end{align*}
\end{definition}

\begin{lemma}[Alternative normal ordering]
  \label{lem:AlternativeNormalOrdering}
  \uses{def:NormalOrdering}
  \lean{VirasoroProject.heiOper_pairNO_eq_pairNO'}
  \leanok
  Suppose that $(\Joper_{k})_{k \in \bZ}$ satisfy the commutation relations
  \tagHeiComm{}. Then for any $k,l \in \bZ$ we have
  \begin{align*}
    \normalOrder{\Joper_k \, \Joper_l} \; = \;
    \begin{cases}
      \Joper_k \circ \Joper_l & \text{ if } k < 0 \\
      \Joper_l \circ \Joper_k & \text{ if } k \ge 0 .
    \end{cases}
  \end{align*}
\end{lemma}
\begin{proof}
  % \uses{}
  \leanok
  Straightforward using the commutation relations \tagHeiComm{}.
\end{proof}

\begin{lemma}[Local truncation for normal ordered products]
  \label{lem:NormalOrderingTruncation}
  \uses{def:NormalOrdering}
  \lean{VirasoroProject.pairNO_apply_eq_zero}
  \leanok
  Suppose that $(\Joper_{k})_{k \in \bZ}$ satisfy the local truncation condition
  \tagHeiTrunc{}. Then for any $v \in V$ there exists an $N$ such that
  whenever $\max \{k,l\} \ge N$ we have
  $\normalOrder{\Joper_{k} \, \Joper_{l}} \, v = 0$.
\end{lemma}
\begin{proof}
  % \uses{}
  \leanok
  Fixing $v \in V$, the local truncation condition \tagHeiTrunc{}
  gives the existence of an $N$ such that $\Joper_{k} \, v = 0$
  for $k \ge N$. It is then clear by inspection of Definition~\ref{def:NormalOrdering}
  that $\normalOrder{\Joper_{k} \, \Joper_{l}} \, v = 0$ when
  $\max \{k,l\} \ge N$.
\end{proof}

\begin{lemma}[Local finite support for homogeneous normal ordered products]
  \label{lem:NormalOrderingFiniteSupport}
  \uses{def:NormalOrdering}
  \lean{VirasoroProject.finite_support_smul_pairNO_heiOper_apply}
  \leanok
  Suppose that $(\Joper_{k})_{k \in \bZ}$ satisfy the local truncation condition
  \tagHeiTrunc{}. Then for any $n \in \bZ$ and any $v \in V$,
  there are only finitely many $k \in \bZ$ such that
  $\normalOrder{\Joper_{n-k} \, \Joper_k} \, v \ne 0$.
\end{lemma}
\begin{proof}
  \uses{lem:NormalOrderingTruncation}
  \leanok
  Straightforward from Lemma~\ref{lem:NormalOrderingTruncation}.
\end{proof}

\begin{definition}[Sugawara operators]
  \label{def:SugawaraOperator}
  \uses{def:NormalOrdering, lem:NormalOrderingFiniteSupport}
  \lean{VirasoroProject.sugawaraGen}
  \leanok
  Suppose that $(\Joper_{k})_{k \in \bZ}$ satisfy the local truncation condition
  \tagHeiTrunc{}. Then for any $n \in \bZ$, a linear operator
  \begin{align*}
    \Loper_n \colon V \to V
  \end{align*}
  can be defined by the formula
  \begin{align*}
    \Loper_n \, v := \frac{1}{2} \sum_{k \in \bZ} \normalOrder{\Joper_{n-k} \, \Joper_k} \, v
    \qquad \text{ for } v \in V
  \end{align*}
  (the sum only has finitely many terms by Lemma~\ref{lem:NormalOrderingFiniteSupport}).

  We call the operators $(\Loper_n)_{n \in \bZ}$ the \term{Sugawara operators}.
\end{definition}

\begin{lemma}[Commutators of Sugawara operators as series]
  \label{lem:SugawaraCommutatorSeries}
  \uses{def:SugawaraOperator}
  \lean{VirasoroProject.commutator_sugawaraGen_apply_eq_finsum_commutator_apply}
  \leanok
  Suppose that $(\Joper_{k})_{k \in \bZ}$ satisfy the local truncation condition
  \tagHeiTrunc{}, and suppose that $\mathsf{A} \colon V \to V$ is a linear operator.
  Then for any $n \in \bZ$, the action of the commutator $[\Loper_n, \mathsf{A}]$
  on any $v \in V$ is given by the series
  \begin{align*}
    [\Loper_n, \mathsf{A}] \, v =
    \frac{1}{2} \sum_{k \in \bZ} [\normalOrder{\Joper_{n-k} \, \Joper_k}, \mathsf{A}] \, v
  \end{align*}
  where only finitely many of the terms are nonzero.
\end{lemma}
\begin{proof}
  \uses{lem:NormalOrderingFiniteSupport}
  \leanok
  Write
  \begin{align*}
    [\Loper_n, A] \, v
    = \; & \Loper_n \, A \, v - A \, \Loper_n \, v \\
    = \; & \frac{1}{2} \sum_{k \in \bZ} \normalOrder{\Joper_{n-k} \, \Joper_k} \, \mathsf{A} \, v
        - \frac{1}{2} \mathsf{A} \sum_{k \in \bZ} \normalOrder{\Joper_{n-k} \, \Joper_k} \, v .
  \end{align*}
  By Lemma~\ref{lem:NormalOrderingFiniteSupport}, only finitely many of the terms
  in both sums are nonzero and they may be rearranged to the asserted form of sum of
  commutators. The resulting sum only has finitely many nonzero terms and is therefore
  well-defined.
\end{proof}

\begin{lemma}[Commutator of Sugawara operators with Heisenberg operators]
  \label{lem:CommutatorSugawaraHeisenberg}
  \uses{def:SugawaraOperator}
  \lean{VirasoroProject.commutator_sugawaraGen_heiOper}
  \leanok
  Suppose that $(\Joper_{k})_{k \in \bZ}$ satisfy the commutation relations
  \tagHeiComm{} and the local truncation condition
  \tagHeiTrunc{}. Then for any $n \in \bZ$ and $k \in \bZ$, we have
  \begin{align*}
    [\Loper_n, \Joper_k] \; = \;
    - k \, \Joper_{n+k} .
  \end{align*}
\end{lemma}
\begin{proof}
  \uses{lem:SugawaraCommutatorSeries}
  \leanok
  Calculation, using Lemma~\ref{lem:SugawaraCommutatorSeries} and
  the commutator formula $[A,BC] = B[A,C] + [A,B]C$.
\end{proof}

\begin{lemma}[Commutator of Sugawara operators with normal ordered pairs]
  \label{lem:CommutatorSugawaraNormalOrderedPair}
  \uses{def:SugawaraOperator}
  \lean{VirasoroProject.commutator_sugawaraGen_heiPairNO'}
  \leanok
  Suppose that $(\Joper_{k})_{k \in \bZ}$ satisfy the commutation relations
  \tagHeiComm{} and the local truncation condition
  \tagHeiTrunc{}. Then for any $n \in \bZ$ and $k, m \in \bZ$, we have
  \begin{align*}
    [\Loper_n, \normalOrder{\Joper_k \, \Joper_{m-k}}]
    = \; &
    -k \, \normalOrder{\Joper_{n+k} \, \Joper_{m-k}}
    - (m-k) \, \normalOrder{\Joper_{k} \, \Joper_{n+m-k}} \\
      & \; + (n+k) \, \delta_{n+m,0}
        \Big( \indicator{-n \le k < 0} - \indicator{0 \le k < - n} \Big) \, \id_V .
  \end{align*}
  where $\indicator{\mathrm{condition}}$ is defined as $1$ if the condition is true
  and $0$ otherwise.
\end{lemma}
\begin{proof}
  \uses{lem:CommutatorSugawaraHeisenberg, lem:AlternativeNormalOrdering}
  \leanok
  Calculation, using Lemmas~\ref{lem:SugawaraCommutatorSeries},
  \ref{lem:AlternativeNormalOrdering}, and~\ref{lem:CommutatorSugawaraHeisenberg},
  the commutation relations \tagHeiComm{},
  and the commutator formula $[A,BC] = B[A,C] + [A,B]C$ again.
\end{proof}

\begin{lemma}[Auxiliary calculation]
  \label{lem:AuxiliaryCentralChargeCalculation}
  %\uses{}
  \lean{VirasoroProject.bosonic_sugawara_cc_calc_nonneg}
  \leanok
  For any $n \in \bN$, we have
  \begin{align*}
    \sum_{l=0}^{n-1} (n-l) l = \frac{n^3 - n}{6} .
  \end{align*}
\end{lemma}
\begin{proof}
  % \uses{}
  \leanok
  Calculation (with induction).
\end{proof}

\begin{lemma}[Virasoro commutation relations for Sugawara operators]
  \label{lem:CommutatorSugawara}
  \uses{def:SugawaraOperator}
  \lean{VirasoroProject.commutator_sugawaraGen}
  \leanok
  Suppose that $(\Joper_{k})_{k \in \bZ}$ satisfy the commutation relations
  \tagHeiComm{} and the local truncation condition
  \tagHeiTrunc{}. Then for any $n, m \in \bZ$, we have
  \begin{align*}
    [\Loper_n, \Loper_m]
    = \; &
    (n-m) \, \Loper_{n+m} + \delta_{n+m,0} \frac{n^3 - n}{12} \, \id_V .
  \end{align*}
\end{lemma}
\begin{proof}
  \uses{lem:CommutatorSugawaraNormalOrderedPair, lem:AuxiliaryCentralChargeCalculation}
  \leanok
  Calculation, using Lemmas~\ref{lem:CommutatorSugawaraNormalOrderedPair}
  and \ref{lem:AuxiliaryCentralChargeCalculation}, among other observations.
\end{proof}

\begin{theorem}[Sugawara construction]
  \label{thm:SugawaraRepresentation}
  \uses{def:SugawaraOperator, def:VirasoroBasis}
  \lean{VirasoroProject.sugawaraRepresentation}
  \leanok
  Suppose that $(\Joper_{k})_{k \in \bZ}$ satisfy the commutation relations
  \tagHeiComm{} and the local truncation condition
  \tagHeiTrunc{}. Then there exists a representation of the Virasoro algebra $\vir$
  with central charge $c = 1$ on $V$ (i.e.,
  the central element $C \in \vir$ acts as $c \, \id_V$ with $c = 1$)
  where the basis elements $L_n$ of $\vir$ act by
  the Sugawara operators $(\Loper_n)_{n \in \bZ}$.
\end{theorem}
\begin{proof}
  \uses{lem:CommutatorSugawara}
  \leanok
  A direct consequence of the commutation relations in Lemma~\ref{lem:CommutatorSugawara}
  and a comparison with the Lie brackets in the basis of Definition~\ref{def:VirasoroBasis}.
\end{proof}
