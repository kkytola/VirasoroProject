\section{A generalized notion of Verma modules}

\begin{definition}[Generalized Verma module]
  \label{def:VermaModule}
  \lean{VirasoroProject.VermaModule, VirasoroProject.instModuleVermaModule}
  \leanok
  Let $A$ be an algebra over a commutative ring $\bbk$.
  Let $\eta = (a_i, r_i)_{i \in I} \in (A \times \bbk)^I$
  be a family of pairs of algebra elements $a_i \in A$ and
  scalars $r_i \in \bbk$.
  The generalized \term{Verma module} associated to the data $\eta$
  is the quotient
  \begin{align*}
    \Verma^{\eta} = A / J^{\eta} ,
  \end{align*}
  where $J^{\eta} \subset A$  is the left ideal (an $A$-submodule) of $A$
  generated by the elements $a_i - r_i \, 1 \in A$.
  Note that $\Verma^{\eta}$ is a (left) $A$-module.
\end{definition}

\begin{lemma}[The highest weight vector of a Verma module]
  \label{def:HighestWeightVector}
  \uses{def:VermaModule}
  \lean{VirasoroProject.VermaModule.hwVec, VirasoroProject.VermaModule.hwVec_cyclic,
    VirasoroProject.VermaModule.apply_hwVec_eq}
  \leanok
  The vector $\VermaHWV^{\eta} := 1 + J^{\eta} \in \Verma^{\eta}$
  (the equivalence class of the unit element of $A$)
  is called the \term{highest weight vector} of $\Verma^{\eta}$.
  It is cyclic, i.e., it generates the whole Verma module $\VermaHWV^{\eta}$
  as an $A$-module, and it satisfies
  \begin{align*}
    a_i \cdot \VermaHWV^{\eta} = r_i \, \VermaHWV^{\eta}
  \end{align*}
  for all $i \in I$, where $a_i \in A$ and $r_i \in \bbk$
  are the algebra elements and scalars in the collection
  $\eta = (a_i, r_i)_{i \in I}$.
\end{lemma}
\begin{proof}
  % \uses{}
  \leanok
  Clear from the construction.
\end{proof}

\begin{lemma}[Universal property of the Verma module]
  \label{lem:VermaUniversalProperty}
  \uses{def:VermaModule, def:HighestWeightVector}
  \lean{VirasoroProject.VermaModule.universalMap, VirasoroProject.VermaModule.universalMap_hwVec,
    VirasoroProject.VermaModule.range_universalMap_eq_span}
  \leanok
  Suppose that $M$ is an $A$-module and $v \in M$ is a vector such that
  \begin{align*}
    a_i \cdot v = r_i \, v
  \end{align*}
  for all $i \in I$, where $a_i \in A$ and $r_i \in \bbk$ are the algebra elements and scalars in
  the collection $\eta = (a_i, r_i)_{i \in I}$. Then there exists a (unique)
  $A$-module homomorphism $\phi \colon \Verma^{\eta} \to M$ such that
  $\phi(\VermaHWV^{\eta}) = v$. The range of the map $\phi$ is the submodule
  generated by $v$ in $M$.
\end{lemma}
\begin{proof}
  % \uses{}
  \leanok
  By cyclicity of the highest weight vector $\VermaHWV^{\eta}$,
  any element of $\Verma^{\eta}$ can be written as $a \cdot \VermaHWV^{\eta}$
  for some $a \in A$.
  The map $A \to M$ given by $a \mapsto a \cdot v$ factors through
  the quotient by the ideal $J^{\eta}$ by our assumption on $v$,
  and since $\Verma^{\eta} = A / J^{\eta}$, we thus get a well-defined map
  $\phi \colon \Verma^{\eta} \to M$ with the asserted properties.
\end{proof}


\section{Verma modules for Lie algebras with a triangular decomposition}

\begin{definition}[Triangular decomposition of a Lie algebra]
  \label{def:TriangularDecomposition}
  %\uses{}
  \lean{VirasoroProject.TriangularDecomposition}
  \leanok
  Let $\gLie$ be a Lie algebra over $\bbk$.
  A \term{triangular decomposition} of $\gLie$ is a decomposition
  \begin{align*}
    \gLie = \gLie^0 \oplus \gLie^+ \oplus \gLie^-
  \end{align*}
  of $\gLie$ into a vector space direct sum of three vector subspaces:
  %$\gLie^0, \gLie^+, \gLie^- \subseteq \gLie$ called the
  \term{Cartan part (or Cartan subalgebra)} $\gLie^0\subseteq \gLie$,
  the \term{upper part} $\gLie^+ \subseteq \gLie$,
  and the \term{lower part} $\gLie^- \subseteq \gLie$.

  (Note that in this definition we do not yet require
  $\gLie^0, \gLie^+, \gLie^- \subseteq \gLie$ to be Lie subalgebras,
  with the Cartan subalgebra being abelian and the
  upper and lower parts being nilpotent. In intended use cases, we
  typically have these properties, however.)
\end{definition}

\begin{definition}[Verma module]
  \label{def:LieVermaModule}
  \uses{def:TriangularDecomposition, def:VermaModule}
  \lean{VirasoroProject.TriangularDecomposition.VermaHW,
    VirasoroProject.TriangularDecomposition.instModule𝓤VermaHW,
    VirasoroProject.TriangularDecomposition.instIsScalarTower𝓤VermaHW}
  \leanok
  Let $\gLie$ be a Lie algebra over $\bbk$, with a triangular decomposition
  $\gLie = \gLie^0 \oplus \gLie^+ \oplus \gLie^-$.
  Let $\eta \colon \gLie^0 \to \bbk$ be a linear functional on the Cartan
  part. Then the \term{Verma module} associated to the "highest weight"
  $\eta$ is the generalized Verma module for the universal enveloping
  algebra $\UEA(\gLie)$, associated with the data
  consisting of the pairs
  $(H, \eta(H)) \in \UEA(\gLie) \times \bbk$ for $H \in \gLie^0$ and
  pairs $(E,0) \in \UEA(\gLie) \times \bbk$ for $E \in \gLie^+$.
  The Verma module is (by a mild abuse of notation)
  still denoted by $\Verma^{\eta}$.
\end{definition}


\section{Virasoro Verma modules}

\begin{definition}[Triangular decomposition of the Virasoro algebra]
  \label{def:VirasoroTriangular}
  \uses{def:TriangularDecomposition, def:VirasoroBasis}
  \lean{VirasoroProject.virasoroTri}
  \leanok
  A triangular decomposition
  \begin{align*}
    \vir = \vir^0 \oplus \vir^+ \oplus \vir^-
  \end{align*}
  of $\vir$ is defined so that $\vir^0$ is spanned by $L_0, C \in \vir$,
  $\vir^+$ is spanned by $L_n$ for $n > 0$,
  and $\vir^-$ is spanned by $L_n$ for $n < 0$.

  (Without further comment, for the Virasoro algebra we always use this
  triangular decomposition.)
\end{definition}

\begin{definition}[Virasoro Verma module]
  \label{def:VirasoroVermaModule}
  \uses{def:VirasoroTriangular, def:LieVermaModule}
  \lean{VirasoroProject.VirasoroVerma}
  \leanok
  Let $c, h \in \bbk$.
  The Virasoro Verma module with central charge $c$ and conformal weight $h$
  is the Verma module $\Verma^{\eta}$ associated to the linear functional
  $\eta \colon \vir^0 \to \bbk$ with
  \begin{align*}
    \eta(L_0) = h, \quad \eta(C) = c .
  \end{align*}
  We denote the Virasoro Verma module by $\Verma^{c,h}$.
\end{definition}
