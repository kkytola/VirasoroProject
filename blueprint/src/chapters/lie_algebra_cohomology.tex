%\section{Lie algebra cohomology in degree two}

Let $\bbk$ be a field and let $\gLie$ be a Lie algebra over $\bbk$.
Fix also a vector space $\aLie$ over $\bbk$, (interpreted, when necessary,
as an abelian Lie algebra, i.e., all Lie brackets in $\aLie$ are taken to
be zero).

\begin{definition}[Lie algebra 1-cocycle]
  \label{def:LieOneCocycle}
  %\uses{}
  \lean{VirasoroProject.LieOneCocycle}
  \leanok
  A \term{1-cocycle} of the Lie algebra $\gLie$ with coefficients in the vector
  space $\aLie$ is a linear map
  \begin{align*}
    \beta \colon \gLie \to \aLie .
  \end{align*}
  The set of all such 1-cocycles is denoted by $\Coc^1(\gLie, \aLie)$.
  %\begin{align*}
  %  \Coc^1(\gLie, \aLie) .
  %\end{align*}
\end{definition}

\begin{definition}[Lie algebra 2-cocycle]
  \label{def:LieTwoCocycle}
  %\uses{}
  \lean{VirasoroProject.LieTwoCocycle}
  \leanok
  A \term{2-cocycle} of the Lie algebra $\gLie$ with coefficients in the vector
  space $\aLie$ is a bilinear map
  \begin{align*}
    \gamma \colon \gLie \times \gLie \to \aLie
  \end{align*}
  such that for all $X \in \gLie$ we have the antisymmetry condition
  \begin{align}\label{eq:LieTwoCocycle.self}
    \gamma(X,X) = \; & 0
  \end{align}
  and for all $X,Y,Z \in \gLie$ we have the Leibnitz rule
  \begin{align}\label{eq:LieTwoCocycle.leibniz}
    \gamma(X,[Y,Z]) = \; & \gamma([X,Y],Z) + \gamma(Y,[X,Z]) .
  \end{align}

  The set of all such 2-cocycles is denoted by $\Coc^2(\gLie, \aLie)$.
  %\begin{align*}
  %  \Coc^1(\gLie, \aLie) .
  %\end{align*}
\end{definition}

\begin{lemma}[Skew-symmetry of 2-cocycles]
  \label{lem:LieTwoCocycle_skew_symmetry}
  \uses{def:LieTwoCocycle}
  \lean{VirasoroProject.LieTwoCocycle.skew}
  \leanok
  For any $\gamma \in \Coc^2(\gLie,\aLie)$ and any $X,Y \in \gLie$,
  we have the skew-symmetry property
  \begin{align*}
    \gamma(X,Y) = - \gamma(Y,X) .
  \end{align*}
\end{lemma}
\begin{proof}
  % \uses{}
  \leanok
  The Leibnitz rule \eqref{eq:LieTwoCocycle.leibniz}
  applied to $X+Y$ gives
  \begin{align*}
    0 = \; & \gamma(X+Y,X+Y) \\
      = \; & \gamma(X,X) + \gamma(X,Y) + \gamma(Y,X) + \gamma(Y,Y)
  \end{align*}
  by bilinearity of $\gamma$. The first and the last terms
  in the last expression vanish by antisymmetry \eqref{eq:LieTwoCocycle.self},
  and the asserted skew-symmetry equation follows.
\end{proof}

\begin{lemma}[Lie algebra 1-cocycles form a vector space]
  \label{lem:LieOneCocycle_vectorSpace}
  \uses{def:LieOneCocycle}
  \lean{VirasoroProject.LieOneCocycle.instModule}
  \leanok
  The set $\Coc^1(\gLie, \aLie)$ of 1-cocycles of $\gLie$ with
  coefficients in $\aLie$ forms a vector space over $\bbk$.
\end{lemma}
\begin{proof}
  % \uses{}
  \leanok
  By definition, $\Coc^1(\gLie, \aLie)$ is the space of
  linear maps $\gLie \to \aLie$, and such linear maps form
  a vector space.
\end{proof}

\begin{lemma}[Lie algebra 2-cocycles form a vector space]
  \label{lem:LieTwoCocycle_vectorSpace}
  \uses{def:LieTwoCocycle}
  \lean{VirasoroProject.LieTwoCocycle.instModule}
  \leanok
  The set $\Coc^2(\gLie, \aLie)$ of 2-cocycles of $\gLie$ with
  coefficients in $\aLie$ forms a vector space over $\bbk$.
\end{lemma}
\begin{proof}
  % \uses{}
  \leanok
  The conditions defining $\Coc^2(\gLie, \aLie)$ are linear,
  so this is staightforward.
\end{proof}

\begin{definition}[Lie algebra 2-coboundary]
  \label{def:LieTwoCoboundary}
  \uses{lem:LieTwoCocycle_vectorSpace, lem:LieOneCocycle_vectorSpace}
  \lean{VirasoroProject.LieOneCocycle.bdryHom}
  \leanok
  Given a 1-cocycle $\beta \in \Coc^1(\gLie, \aLie)$, we define the
  \term{coboundary} $\partial \beta$ of $\beta$ to be the bilinear map
  \begin{align*}
    \partial \beta \colon \; & \gLie \times \gLie \to \aLie \\
  \end{align*}
  given by
  \begin{align*}
    \partial \beta (X, Y) = \; & \beta ([X,Y]) .
  \end{align*}
  We then have $\partial \beta \in \Coc^2(\gLie,\aLie)$.
  The mapping $\partial \colon \Coc^1(\gLie,\aLie) \to \Coc^2(\gLie,\aLie)$
  is linear. Its range is denoted $\Cob^2(\gLie,\aLie) \subset \Coc^2(\gLie,\aLie)$
  and called the set of \term{2-coboundaries} of the Lie algebra $\gLie$
  with coefficients in $\aLie$.
\end{definition}

\begin{definition}[Lie algebra 2-cohomology]
  \label{def:LieTwoCohomology}
  \uses{def:LieTwoCoboundary}
  \lean{VirasoroProject.LieTwoCohomology}
  \leanok
  The vector space
  \begin{align*}
    \Coh^2(\gLie,\aLie) := \Coc^2(\gLie,\aLie) \, / \, \Cob^2(\gLie,\aLie)
  \end{align*}
  is called the \term{Lie algebra cohomology in degree 2} of $\gLie$
  with coefficients in $\aLie$.
\end{definition}

\begin{lemma}[Cohomology of abelian Lie algebras]
  \label{lem:LieTwoCohomology_abelian}
  \uses{def:LieTwoCohomology}
  \lean{VirasoroProject.LieTwoCocycle.toLieTwoCohomologyEquiv_toLinearMap}
  \leanok
  If $\gLie$ is abelian, i.e., $[\gLie,\gLie] = 0$, then
  the canonical projection
  \begin{align*}
    \Coc^2(\gLie,\aLie) \to \Coh^2(\gLie,\aLie)
  \end{align*}
  %mapping a 2-cocycle $\gamma$ to its equivalence class
  %$\gamma + \Cob^2(\gLie,\aLie)$
  is a linear isomorphism.
\end{lemma}
\begin{proof}
  % \uses{}
  \leanok
  The projection is surjective by construction, so it suffices to show that it is also injective.
  The kernel of the projection is $\Cob^2(\gLie,\aLie) = \Ima \; \partial$.
  In view of Definition~\ref{def:LieTwoCoboundary},
  abelianity of $\gLie$ implies $\partial \beta = 0$ for any $\beta \in \Coc^1(\gLie,\aLie)$.
  Therefore $\Cob^2(\gLie,\aLie) = 0$, and the kernel of the projection is trivial,
  so the projection is indeed injective.
\end{proof}
