In this section we assume that $\bbk$ is a field of characteristic zero.
% and $\gLie$ is the vector space with basis $(j_k)_{k \in \bZ}$ over $\bbk$
% considered as an abelian Lie algebra.

\begin{definition}[Heisenberg cocycle]
  \label{def:HeisenbergCocycle}
  \uses{def:LieTwoCocycle}
  \lean{VirasoroProject.AbelianLieAlgebraOn.heisenbergCocycle}
  \leanok
  Let $\gLie$ be the vector space with basis $(j_k)_{k \in \bZ}$ over $\bbk$,
  considered as an abelian Lie algebra.
  The bilinear map ${\gamma}_{\hei} \colon \gLie \times \gLie \to \bbk$
  given on basis elements by
  \begin{align*}
    {\gamma}_{\hei}(j_k, j_l) = k \, \delta_{k+l,0}
  \end{align*}
  is a Lie algebra 2-cocycle, ${\gamma}_{\hei} \in C^2(\gLie,\bbk)$.
  We call ${\gamma}_{\hei}$ the \term{Heisenberg cocycle}.
\end{definition}

\begin{lemma}[The Heisenberg cocyle is nontrivial]
  \label{lem:HeisenbergCocycleNontrivial}
  \uses{def:HeisenbergCocycle, def:LieTwoCohomology}
  \lean{VirasoroProject.AbelianLieAlgebraOn.heisenbergCocycle_ne_zero}
  \leanok
  The cohomology class $[{\gamma}_{\hei}] \in H^2(\gLie,\bbk)$
  of the Heisenberg cocycle is nonzero.
\end{lemma}
\begin{proof}
  % \uses{}
  \leanok
  \ldots
\end{proof}

\begin{definition}[Heisenberg algebra]
  \label{def:HeisenbergAlgebra}
  \uses{def:HeisenbergCocycle, def:CentralExtensionOfCocycle}
  \lean{VirasoroProject.HeisenbergAlgebra, VirasoroProject.HeisenbergAlgebra.instLieAlgebra,
  VirasoroProject.HeisenbergAlgebra.isCentralExtension}
  \leanok
  Let $\bbk$ be a field of characteristic zero.
  The \term{Heisenberg algebra} $\hei$ is the Lie algebra over $\bbk$
  obtained as the central extension of the abelian Lie algebra $\gLie$
  with basis $(j_k)_{k \in \bZ}$,
  corresponding to the Heisenberg cocycle ${\gamma}_{\hei} \in C^2(\gLie,\bbk)$.
\end{definition}
