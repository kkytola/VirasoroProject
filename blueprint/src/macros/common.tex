% In this file you should put all LaTeX macros and settings to be used both by
% the pdf version and the web version.
% This should be most of your macros.

% The theorem-like environments defined below are those that appear by default
% in the dependency graph. See the README of leanblueprint if you need help to
% customize this.
% The configuration below use the theorem counter for all those environments
% (this is what the [theorem] arguments mean) and never resets it.
% If you want for instance to number them within chapters then you can add
% [chapter] at the end of the next line.
\newtheorem{theorem}{Theorem}
\newtheorem{proposition}[theorem]{Proposition}
\newtheorem{lemma}[theorem]{Lemma}
\newtheorem{corollary}[theorem]{Corollary}

\theoremstyle{definition}
\newtheorem{definition}[theorem]{Definition}

\newcommand{\term}[1]{\textbf{#1}}

\newcommand{\bN}{\mathbb{N}}
\newcommand{\bZ}{\mathbb{Z}}
\newcommand{\bQ}{\mathbb{Q}}
\newcommand{\bR}{\mathbb{R}}
\newcommand{\bC}{\mathbb{C}}

\newcommand{\bbk}{\mathbb{K}}
\newcommand{\gLie}{\mathfrak{g}}
%\newcommand{\hLie}{\mathfrak{h}}
%\newcommand{\hLie}{\mathfrak{h}}
\newcommand{\aLie}{\mathfrak{a}}
\newcommand{\egLie}{\mathfrak{h}}
\newcommand{\hei}{\mathfrak{hei}}
\newcommand{\witt}{\mathfrak{witt}}
\newcommand{\vir}{\mathfrak{vir}}

\newcommand{\Joper}{\mathsf{J}}
\newcommand{\Loper}{\mathsf{L}}
\newcommand{\tagHeiComm}{\textrm{(HeiComm)}}
\newcommand{\tagHeiTrunc}{\textrm{(HeiTrunc)}}
\newcommand{\normalOrder}[1]{{\mathbb{:} #1 \mathbb{:}}}

\newcommand{\id}{\mathrm{id}}
\newcommand{\idOf}[1]{\id_{{#1}}}

\newcommand{\indicator}[1]{\mathbb{I}_{{#1}}}

\newcommand{\Ima}{\mathrm{Im}}
\newcommand{\Ker}{\mathrm{Ker}}
